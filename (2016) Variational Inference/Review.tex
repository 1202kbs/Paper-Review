\documentclass[10pt]{article}

\usepackage{mathtools}
\usepackage{fullpage}
\usepackage{latexsym}
\usepackage{csquotes}
\usepackage{enumitem}
\usepackage{mathrsfs}
\usepackage{hyperref}
\usepackage{tikz-cd}
\usepackage{amsmath}
\usepackage{amssymb}
\usepackage{gensymb}
\usepackage{parskip}
\usepackage{xcolor}
\usepackage{amsthm}
\usepackage{tikz}
\usepackage{bm}

\newenvironment{solution}{\textit{Solution.}}{\hfill$\square$}


\newcommand{\EE}{\mathbb{E}}

\newcommand{\CC}{\mathcal{C}}
\newcommand{\FF}{\mathcal{F}}
\newcommand{\HH}{\mathcal{H}}
\newcommand{\KK}{\mathcal{K}}
\newcommand{\LL}{\mathcal{L}}
\newcommand{\MM}{\mathcal{M}}
\newcommand{\NN}{\mathcal{N}}
\newcommand{\OO}{\mathcal{O}}
\newcommand{\QQ}{\mathcal{Q}}

\newcommand{\lbo}{\texttt{LBO}}
\newcommand{\elbo}{\texttt{ELBO}}
\newcommand{\KL}{D_{\text{KL}}}
\newcommand{\JS}{D_{\text{JS}}}
\newcommand{\stopg}{\texttt{stop\char`_grad}}

\DeclareMathOperator*{\argmax}{arg\,max}
\DeclareMathOperator*{\argmin}{arg\,min}


\title{Paper Review: Variational Inference}
\author{Beomsu Kim\footnote{Department of Mathematical Sciences, KAIST. Email \texttt{beomsu.kim@kaist.ac.kr}}}
\date{\today}

\begin{document}

\maketitle

\textbf{Paper Information.}

\begin{itemize}
\item David M. Blei et. al. Variational Inference: A Review for Statisticians. \\ \underline{Journal of the American Statistical Association}, 112(518):859--877, 2017.
\end{itemize}

\section{Introduction}

\begin{itemize}
\item A core problem of modern statistics is to approximate difficult-to-compute probability densities.
\item Consider a joint density of latent variables $\bm{z} = z_{1:m}$ and observations $\bm{x} = x_{1:n}$
\begin{align*}
p(\bm{z},\bm{x}) = p(\bm{z}) p(\bm{x} \mid \bm{z}).
\end{align*}
A Bayesian model draws the latent variables from a prior density $p(\bm{z})$ and then relates them back to the observations through the likelihood $p(\bm{x} \mid \bm{z})$.
\item Inference amounts to conditioning on the data and computing the posterior $p(\bm{z} \mid \bm{x})$.
\item In complex or high-dimensional Bayesian models, this computation is often intractable.
\item There are two approaches to approximate inference: MCMC and variational inference.
\begin{itemize}
\item MCMC first constructs an ergodic Markov chain on $\bm{z}$ whose stationary distribution is the posterior $p(\bm{z} \mid \bm{x})$. Then, we sample from the chain to collect samples from the stationary distribution.
\item Variational inference uses a family of tractable\footnote{A distribution is \textit{tractable} if it has a closed form density function or we can easily sample from it.} distributions $\QQ$ to approximate $p(\bm{z} \mid \bm{x})$. Equivalently, variational inference uses $\QQ$ to approximate $p(\bm{x})$.\footnote{Approximating $p(\bm{z} \mid \bm{x})$ is equivalent to approximating $p(\bm{x})$ since $p(\bm{z} \mid \bm{x}) = p(\bm{z},\bm{x}) / p(\bm{x})$.}
\end{itemize}
\item Comparing variational inference and MCMC.
\begin{itemize}
\item MCMC methods tend to be more computationally expensive than variational inference, but they also provide guarantees of producing (asymptotically) exact samples from the target density.
\item Variational inference does not enjoy such guarantees---it can only find a density close to the target---but tends to be faster than MCMC. Because it rests on optimization, variational inference easily takes advantage of methods like stochastic optimization and distributed optimization.
\end{itemize}
\item In the following sections, I omit examples for clarity of exposition. Please read the reference materials for detailed examples.
\end{itemize}

\newpage

\section{Variational Inference}

\begin{itemize}
\item Let $\bm{x} = x_{1:n}$ be a set of observed variables and $\bm{z} = z_{1:m}$ be a set of latent variables with joint density
\begin{align*}
p(\bm{z},\bm{x}).
\end{align*}
\item The \textit{inference problem} is to compute the conditional density of $\bm{z}$ given $\bm{x}$
\begin{align*}
p(\bm{z} \mid \bm{x}) = \frac{p(\bm{z},\bm{x})}{p(\bm{x})}.
\end{align*}
The denominator is called the \textit{evidence}. We calculate it by marginalizing out the latent variables
\begin{align*}
p(\bm{x}) = \int p(\bm{z},\bm{x}) \, d\bm{z}.
\end{align*}
For many models, the evidence integral is unavailable in closed form or requires exponential (w.r.t. the dimension $n$) time to compute. This is why inference in such models is hard.
\item Hence, we resort to approximate inference. There are two equivalent approaches.
\end{itemize}

\subsection{Approach 1: Evidence Lower Bound (\elbo)} \label{sec:elbo}

\begin{itemize}
\item Assuming we have $p(\bm{z},\bm{x})$, calculating $p(\bm{z} \mid \bm{x})$ is equivalent to calculating $p(\bm{x})$.
\item Instead of directly calculating $p(\bm{x})$, we maximize a lower bound of $p(\bm{x})$.
\item Specifically, we first define a \textit{variational family} $\QQ$ of tractable densities over the latent variables.
\item Then, for any $q(\bm{z}) \in \QQ$,
\begin{align*}
\log p(\bm{x}) &= \log \int p(\bm{z},\bm{x}) \, d\bm{z} \\
&= \log \EE_{q(\bm{z})} \left[ \frac{p(\bm{z},\bm{x})}{q(\bm{z})} \right] \\
&\geq \EE_{q(\bm{z})} \left[ \log \frac{p(\bm{z},\bm{x})}{q(\bm{z})} \right] \\
&= \EE_{q(\bm{z})} [ \log p(\bm{z},\bm{x}) ] - \EE_{q(\bm{z})} [\log q(\bm{z})] \\
&= \elbo[q]
\end{align*}
where we have used Jensen's inequality at the fourth line. $\elbo$ is defined as
\begin{align} \label{eq:elbo1}
\elbo[q] = \EE_{q(\bm{z})} [ \log p(\bm{z},\bm{x}) ] - \EE_{q(\bm{z})} [\log q(\bm{z})]
\end{align}
and it lower bounds the (log) evidence $p(\bm{x})$. From here comes its name \enquote{evidence lower bound}.
\item We can also obtain $\elbo$ by the following process.
\begin{align} \label{eq:elbomax}
\log p(\bm{x}) &= \int q(\bm{z}) \log p(\bm{x}) \, d\bm{z} \nonumber \\
&= \int q(\bm{z}) \log \frac{p(\bm{z},\bm{x})}{p(\bm{z} \mid \bm{x})} \, d\bm{z} \nonumber \\
&= \int q(\bm{z}) \log \frac{p(\bm{z},\bm{x})}{q(\bm{z})} \cdot \frac{q(\bm{z})}{p(\bm{z} \mid \bm{x})} \, d\bm{z} \nonumber \\
&= \EE_{q(\bm{z})} [ \log p(\bm{z},\bm{x}) ] - \EE_{q(\bm{z})} [\log q(\bm{z})] + \KL(q(\bm{z}) \| p(\bm{z} \mid \bm{x})) \nonumber \\
&= \elbo[q] + \KL(q(\bm{z}) \| p(\bm{z} \mid \bm{x})) \\
&\geq \elbo[q]. \nonumber
\end{align}
\item Hence, we can solve the optimization problem
\begin{align*}
\max_{q(\bm{z}) \in \QQ} \ \elbo[q]
\end{align*}
to obtain the best approximation to $\log p(\bm{x})$.
\item Equation \eqref{eq:elbomax} shows that $\elbo$ is maximized when $q(\bm{z}) = p(\bm{z} \mid \bm{x})$.
\item Calculus of variations can also be used to prove that $q(\bm{z}) = p(\bm{z} \mid \bm{x})$ maximizes the $\elbo$.
\item For a detailed proof, see Appendix \ref{append:varelbo}.
\item We also observe that
\begin{align} \label{eq:elbo2}
\elbo[q] &= \EE_{q(\bm{z})} [ \log p(\bm{z},\bm{x}) ] - \EE_{q(\bm{z})} [\log q(\bm{z})] \nonumber \\
&= \EE_{q(\bm{z})}[\log p(\bm{x} \mid \bm{z})] + \EE_{q(\bm{z})}[ \log p(\bm{z}) ] - \EE_{q(\bm{z})} [\log q(\bm{z})] \nonumber \\
&= \EE_{q(\bm{z})}[\log p(\bm{x} \mid \bm{z})] - \KL(q(\bm{z}) \| p(\bm{z})).
\end{align}
\item The first term is an expected likelihood, and it encourages densities that place their mass on configurations of the latent variables that explain the observed data.
\item The second term is the negative divergence between the variational density and the prior; it encourages densities close to the prior.
\end{itemize}

\subsection{Approach 2: Posterior Approximation}

\begin{itemize}
\item We specify a family $\QQ$ of densities over the latent variables.
\item Each candidate $q(\bm{z}) \in \QQ$ is a candidate approximation to the exact conditional $p(\bm{z} \mid \bm{x})$.
\item Inference now amounts to solving the following optimization problem
\begin{align*}
q^*(\bm{z}) &= \argmin_{q(\bm{z}) \in \QQ} \KL(q(\bm{z}) \| p(\bm{z} \mid \bm{x})) \\
&= \argmin_{q(\bm{z}) \in \QQ} \EE_{q(\bm{z})}[ \log q(\bm{z}) ] - \EE_{q(\bm{z})}[\log p(\bm{z} \mid \bm{x})] \\
&= \argmin_{q(\bm{z}) \in \QQ} \EE_{q(\bm{z})}[ \log q(\bm{z}) ] - \EE_{q(\bm{z})}[\log p(\bm{z}, \bm{x})] + \log p(\bm{x}) \\
&= \argmin_{q(\bm{z}) \in \QQ} \EE_{q(\bm{z})}[ \log q(\bm{z}) ] - \EE_{q(\bm{z})}[\log p(\bm{z}, \bm{x})] \\
&= \argmin_{q(\bm{z}) \in \QQ} - \elbo[q] \\
&= \argmax_{q(\bm{z}) \in \QQ} \elbo[q].
\end{align*}
\item Once found, $q^*(\bm{z})$ is the best approximation of the conditional, within the family $\QQ$.
\item The complexity of the family determines the complexity of this optimization.
\item Since
\begin{align*}
\argmin_{q(\bm{z}) \in \QQ} \KL(q(\bm{z}) \| p(\bm{z} \mid \bm{x})) = \argmax_{q(\bm{z}) \in \QQ} \elbo[q],
\end{align*}
approaches 1 and 2 are equivalent.
\end{itemize}

\newpage

\section{Mean-Field Variational Inference}

\begin{itemize}
\item We now know we can do approximate inference by maximizing the $\elbo$ w.r.t. a variational family $\QQ$.
\item We give an example of a variational family $\QQ$ that is often used in the literature.
\item We focus on the \textit{mean-field variational family}, where the latent variables are mutually independent and each governed by a distinct factor in the variational density.
\begin{align} \label{eq:mf}
q(\bm{z}) = \prod_{j = 1}^m q_j(z_j).
\end{align}
Each latent variable $z_j$ is governed by its own variational factor, the density $q_j(z_j)$.
\end{itemize}

\subsection{Coordinate Ascent Mean-Field Variational Inference (CAVI)} \label{sec:cavi}

\begin{itemize}
\item CAVI optimizes each factor of the mean-field variational density, while holding the others fixed.
\item It climbs the $\elbo$ to a local maximum.
\item Define
\begin{align*}
\bm{z}_{-j} = (z_1, \ldots, z_{j - 1},z_{j + 1}, \ldots, z_m), \qquad q_{-j}(\bm{z}_{-j}) = \prod_{\ell \neq j} q_j(z_j).
\end{align*}
\item The \textit{complete conditional} of $z_j$ is its conditional density given all of the other latent variables in the model and the observations $p(z_j \mid \bm{z}_{-j}, \bm{x})$.
\item The CAVI update is given by
\begin{align} \label{eq:cavi1}
q^*_j(z_j) \propto \exp\left\{ \EE_{q_{-j}(\bm{z}_{-j})}[ \log p(z_j \mid \bm{z}_{-j}, \bm{x})] \right\}.
\end{align}
Equivalently, Equation \eqref{eq:cavi1} is proportional to
\begin{align} \label{eq:cavi2}
q^*_j(z_j) \propto \exp\left\{ \EE_{q_{-j}(\bm{z}_{-j})}[ \log p(z_j, \bm{z}_{-j}, \bm{x}) ] \right\}.
\end{align}
Because of the mean-field family assumption, the expectations of on the RHS do not involve the $j$th variational factor. Thus this is a valid coordinate update.
\item We now derive the CAVI update. Specifically, define
\begin{align*}
Z[q_{-j}] = \int \exp\left\{ \EE_{q_{-j}(\bm{z}_{-j})}[ \log p(z_j, \bm{z}_{-j}, \bm{x}) ] \right\} \, dz_j
\end{align*}
such that
\begin{align*}
q_j^*(z_j) = \frac{1}{Z[q_{-j}]} \exp\left\{ \EE_{q_{-j}(\bm{z}_{-j})}[ \log p(z_j, \bm{z}_{-j}, \bm{x}) ] \right\}
\end{align*}
and we rewrite the $\elbo$ as a functional of $q_j$.
\begin{align*}
\elbo[q_j] &= \EE_{q(\bm{z})} [ \log p(\bm{z},\bm{x}) ] - \EE_{q(\bm{z})} [\log q(\bm{z})] \\
&= \EE_{q(\bm{z})} [ \log p(\bm{z},\bm{x}) ] - \EE_{q(\bm{z})} [\log q_j(z)] - \EE_{q(\bm{z})} [\log q_{-j}(\bm{z}_{-j})] \\
&= \EE_{q_j(z_j)} \left[ \EE_{q_{-j}(\bm{z}_{-j})} [ \log p(z_j, \bm{z}_{-j}, \bm{x}) ] \right] - \EE_{q_j(z_j)} [\log q_j(z)] - \EE_{q_{-j}(\bm{z}_{-j})} [\log q_{-j}(\bm{z}_{-j})] \\
&= \EE_{q_j(z_j)} \left[ \log q^*_j(z_j) \right] - \EE_{q_j(z_j)} [\log q_j(z)] - \EE_{q_{-j}(\bm{z}_{-j})} [\log q_{-j}(\bm{z}_{-j})] + Z[q_{-j}] \\
&= -\KL(q^*_j(z_j) \| q_j(z_j)) - \EE_{q_{-j}(\bm{z}_{-j})} [\log q_{-j}(\bm{z}_{-j})] + Z[q_{-j}].
\end{align*}
Since the second and the third terms are constant w.r.t. $q_j$, $\elbo[q_j]$ is maximized when $q_j = q^*_j$.
\item We can also use Calculus of Variations to derive the CAVI update.
\item See \url{http://www2.imm.dtu.dk/pubdb/edoc/imm3314.pdf} for an example.
\end{itemize}

\section{Expectation Maximization (EM)}

\begin{itemize}
\item Let $\bm{x}$ be a set of observed variables and $\bm{y}$ be a set of latent variables with joint density
\begin{align*}
p(\bm{y},\bm{x} \mid \bm{\theta})
\end{align*}
where $\bm{\theta}$ is the set of parameters of $p$.
\end{itemize}

\subsection{EM for MLE}

\begin{itemize}
\item The goal of MLE is to solve
\begin{align*}
\argmax_{\bm{\theta}} \log p(\bm{x} \mid \bm{\theta}).
\end{align*}
\item Since
\begin{align*}
\log p(\bm{x} \mid \bm{\theta}) = \log \int p(\bm{y}, \bm{x} \mid \bm{\theta}) \, d\bm{z},
\end{align*}
the integral appears \textit{inside} the log. This can make optimization difficult.
\item We recall that (c.f. Equation \eqref{eq:elbomax})
\begin{align*}
\log p(\bm{x} \mid \bm{\theta}) = \elbo[q,\bm{\theta}] + \KL(q(\bm{y}) \| p(\bm{y} \mid \bm{x}, \bm{\theta})) \geq \elbo[q,\bm{\theta}]
\end{align*}
where
\begin{align*}
\elbo[q,\bm{\theta}] = \EE_{q(\bm{y})} [ \log p(\bm{y},\bm{x} \mid \bm{\theta}) ] - \EE_{q(\bm{y})} [\log q(\bm{y})].
\end{align*}
Since
\begin{align*}
\EE_{q(\bm{y})} [ \log p(\bm{y},\bm{x} \mid \bm{\theta}) ] = \int q(\bm{y}) \log p(\bm{y},\bm{x} \mid \bm{\theta}) \, d\bm{z},
\end{align*}
the integral appears \textit{outside} the log so the maximization of the $\elbo$ w.r.t. $q$ or $\bm{\theta}$ can be easier.
\item Motivated by this observation, we take a two-step approach.
\begin{itemize}
\item \textbf{E (expectation) step} : hold $\bm{\theta}$ constant and solve 
\begin{align*}
q^*(\bm{y}) = \argmax_{q(\bm{y})} \elbo[q,\bm{\theta}] = p(\bm{y} \mid \bm{x},\bm{\theta}).
\end{align*}
We then calculate the expectation
\begin{align*}
\EE_{q^*(\bm{y})} [ \log p(\bm{y},\bm{x} \mid \bm{\theta}) ] = \EE_{p(\bm{y} \mid \bm{x},\bm{\theta})} [ \log p(\bm{y},\bm{x} \mid \bm{\theta}) ]
\end{align*}
which is the only term in the $\elbo[q^*,\bm{\theta}]$ depending on $\bm{\theta}$.
\item \textbf{M (maximization) step} : hold $q^*$ constant and solve
\begin{align*}
\bm{\theta}^* = \argmax_{\bm{\theta}} \elbo[q^*,\bm{\theta}] = \argmax_{\bm{\theta}} \EE_{p(\bm{y} \mid \bm{x},\bm{\theta})} [ \log p(\bm{y},\bm{x} \mid \bm{\theta}) ].
\end{align*}
\item Set $q$ and $\bm{\theta}$ as $q^*$ and $\bm{\theta}^*$ and repeat the above two steps.
\end{itemize}
\item This indeed increases $\log p(\bm{x} \mid \bm{\theta})$ since
\begin{align*}
\log p(\bm{x} \mid \bm{\theta}) &= \elbo[q,\bm{\theta}] + \KL(q(\bm{y}) \| p(\bm{y} \mid \bm{x}, \bm{\theta})) \\
&= \elbo[q^*,\bm{\theta}] \\
&\leq \elbo[q^*,\bm{\theta}^*] \\
&\leq \elbo[q^*,\bm{\theta}^*] + \KL(q^*(\bm{y}) \| p(\bm{y} \mid \bm{x}, \bm{\theta})) \\
&= \log p(\bm{x} \mid \bm{\theta}^*).
\end{align*}
\end{itemize}

\subsection{EM for MAP Estimation}

\begin{itemize}
\item We introduce a prior distribution for $\bm{\theta}$, denoted $\bm{\theta}$, such that
\begin{align*}
p(\bm{x},\bm{\theta}) = p(\bm{x} \mid \bm{\theta}) p(\bm{\theta}).
\end{align*}
\item The goal of MAP is to solve
\begin{align*}
\argmax_{\bm{\theta}} \log p(\bm{\theta} \mid \bm{x}) = \argmax_{\bm{\theta}} \log p(\bm{x}, \bm{\theta}).
\end{align*}
\item Replacing $\bm{x}$ by $(\bm{x},\bm{\theta})$ in Equation \eqref{eq:elbomax}, we see that
\begin{align*}
\log p(\bm{x},\bm{\theta}) = \lbo[q,\bm{\theta}] + \KL(q(\bm{y}) \| p(\bm{y} \mid \bm{x}, \bm{\theta})) \geq \lbo[q,\bm{\theta}]
\end{align*}
where
\begin{align*}
\lbo[q,\bm{\theta}] = \EE_{q(\bm{y})}[\log p(\bm{y}, \bm{x}, \bm{\theta})] - \EE_{q(\bm{y})}[\log q(\bm{y})].
\end{align*}
I will call the above term just \enquote{lower bound} not $\elbo$, since $p(\bm{x},\bm{\theta})$ is not evidence.
\item We again take a two-step approach.
\begin{itemize}
\item \textbf{E (expectation) step} : hold $\bm{\theta}$ constant and solve 
\begin{align*}
q^*(\bm{y}) = \argmax_{q(\bm{y})} \lbo[q,\bm{\theta}] = p(\bm{y} \mid \bm{x},\bm{\theta}).
\end{align*}
We then calculate the expectation
\begin{align*}
\EE_{q^*(\bm{y})} [ \log p(\bm{y},\bm{x}, \bm{\theta}) ] = \EE_{p(\bm{y} \mid \bm{x},\theta)} [ \log p(\bm{y},\bm{x}, \bm{\theta}) ]
\end{align*}
which is the only term in the $\lbo[q^*,\bm{\theta}]$ depending on $\bm{\theta}$.
\item \textbf{M (maximization) step} : hold $q^*$ constant and solve
\begin{align*}
\bm{\theta}^* = \argmax_{\bm{\theta}} \lbo[q^*,\bm{\theta}] = \argmax_{\bm{\theta}} \EE_{p(\bm{y} \mid \bm{x},\bm{\theta})} [ \log p(\bm{y},\bm{x}, \bm{\theta}) ].
\end{align*}
\item Set $q$ and $\bm{\theta}$ as $q^*$ and $\bm{\theta}^*$ and repeat the above two steps.
\end{itemize}
\item This indeed increases $\log p(\bm{x}, \bm{\theta})$ since
\begin{align*}
\log p(\bm{x}, \bm{\theta}) &= \lbo[q,\bm{\theta}] + \KL(q(\bm{y}) \| p(\bm{z} \mid \bm{x}, \bm{\theta})) \\
&= \lbo[q^*,\bm{\theta}] \\
&\leq \lbo[q^*,\bm{\theta}^*] \\
&\leq \lbo[q^*,\bm{\theta}^*] + \KL(q^*(\bm{y}) \| p(\bm{y} \mid \bm{x}, \bm{\theta})) \\
&= \log p(\bm{x}, \bm{\theta}^*).
\end{align*}
\end{itemize}

\textbf{Remark.}
\begin{itemize}
\item EM maximizes the $\elbo$ exactly while variational inference maximizes the $\elbo$ approximately.
\item EM which uses Monte Carlo approximation in the E step (in this case, E step of EM for MLE)
\begin{align*}
\EE_{p(\bm{y} \mid \bm{x},\bm{\theta})} [ \log p(\bm{y},\bm{x} \mid \bm{\theta}) ] \approx \frac{1}{L} \sum_{\ell = 1}^L \log p(\bm{y}^{(\ell)},\bm{x} \mid \bm{\theta})
\end{align*}
where $\bm{y}^{(\ell)}$ are samples from $p(\bm{y} \mid \bm{x},\bm{\theta})$ is called \textit{Monte Carlo EM}.
\end{itemize}

\textbf{Reference Material.}
\begin{itemize}
\item \textit{Pattern Recognition and Machine Learning} by Christopher M. Bishop.
\end{itemize}

\newpage

\section{Variational EM}

\begin{itemize}
\item Let $\MM$ be a collection of model structures.
\item Each model structure $m \in \MM$ has a set of parameters $\bm{\theta}$.
\item Let $\bm{x}$ be a set of observed variables and $\bm{y}$ be a set of latent variables.
\item For each model structure $m$, we then have the joint distribution
\begin{align*}
p(\bm{x}, \bm{y}, \bm{\theta} \mid m).
\end{align*}
\item Our goal is to calculate the evidence
\begin{align*}
\log p(\bm{x} \mid m) = \log \iint p(\bm{x}, \bm{y}, \bm{\theta} \mid m) \, d\bm{y} \, d\bm{\theta}.
\end{align*}
Then, we can either perform MLE w.r.t. $m$
\begin{align*}
\argmax_{m \in \MM} \log p(\bm{x} \mid m)
\end{align*}
or given a prior distribution over model structures $p(m)$, we can perform MAP w.r.t. $m$
\begin{align*}
\argmax_{m \in \MM} \log p(m \mid \bm{x}) = \argmax_{m \in \MM} \log p(\bm{x},m) = \argmax_{m \in \MM} \log p(\bm{x} \mid m) p(m).
\end{align*}
\item Recall that maximizing the $\elbo$ leads to approximating the log evidence (Section \ref{sec:elbo}).
\item Hence, setting $z_1 = \bm{y}$ and $z_2 = \bm{\theta}$ such that $\bm{z} = (\bm{y},\bm{\theta})$, we may use CAVI (Section \ref{sec:cavi}).
\item Specifically, define the mean-field variational distribution (c.f. Equation \eqref{eq:mf})
\begin{align*}
q(\bm{z}) = q(\bm{y},\bm{\theta}) = q_1(\bm{y}) q_2(\bm{\theta}).
\end{align*}
Equation \eqref{eq:cavi1} gives us the update rule
\begin{align*}
q_1^*(\bm{y}) &\propto \exp\left\{ \EE_{q_2(\bm{\theta})}[\log p(\bm{y} \mid \bm{\theta}, \bm{x}, m)] \right\}, \\
q_2^*(\bm{\theta}) &\propto \exp\left\{ \EE_{q_1^*(\bm{y})}[\log p(\bm{\theta} \mid \bm{y}, \bm{x}, m)] \right\}.
\end{align*}
We then set $q_1(\bm{y})$ and $q_2(\bm{\theta})$ as $q_1^*(\bm{y})$ and $q_2^*(\bm{\theta})$ and repeat the above two steps.
\item Somewhat like EM, variational EM alternates between a $q_1(\bm{y})$ update and a $q_2(\bm{\theta})$ update.
\item This is why it is called variational \enquote{EM}.
\end{itemize}

\textbf{Reference Material.}
\begin{itemize}
\item Matthew J. Beal and Zoubin Ghahramani. \textit{The Variational Bayesian EM Algorithm for Incomplete Data: with Applications to Scoring Graphical Model Structures}. In \underline{Bayesian Statistics 7}, 2003.
\end{itemize}

\newpage

\appendix

\section{Calculus of Variations}

\subsection{Preliminaries}

\begin{itemize}
\item A \textit{functional} is a scalar-valued function defined on the space of functions.
\item Formally, a functional $\FF$, when given a function $u$, returns a scalar $\FF[u]$.
\item The \textit{calculus of variations} is a field of mathematics that uses variations, which are small changes in functions and functionals, to find maxima and minima of functionals.
\item We use the functional gradient, denoted $\nabla \FF[u]$, to find maxima or minima of functionals.
\item We treat the functional gradient as a directional derivative
\begin{align} \label{eq:grad}
\langle \nabla \FF[u], v \rangle = \frac{d}{d\lambda} \FF[u + \lambda v] \bigg\rvert_{\lambda = 0}
\end{align}
where $\lambda \in \mathbb{R}$.
\item The function $v$ representing the direction of the derivative is called the \textit{variation} of the function $u$.
\item The inner product is the standard $L^2$ inner product for real functions
\begin{align*}
\langle f, g \rangle = \int f(x) g(x) \, dx.
\end{align*}
\end{itemize}

\textbf{Proposition 1.} \textit{For a differentiable function $f$, if}
\begin{align*}
\FF[u] = \int f(u(x)) \, dx,
\end{align*}
\textit{then the functional gradient is given by}
\begin{align*}
\nabla \FF[u] = \frac{\partial}{\partial u} f(u) = f'(u).
\end{align*}

\begin{proof}
Observe that
\begin{align*}
\frac{d}{d\lambda} \FF[u + \lambda v] &= \frac{d}{d\lambda} \int f(u(x) + \lambda v(x)) \, dx \\
&= \int \frac{d}{d\lambda} f(u(x) + \lambda v(x)) \, dx \\
&= \int f'(u(x) + \lambda v(x)) v(x) \, dx
\end{align*}
and so
\begin{align*}
\frac{d}{d\lambda} \FF[u + \lambda v] \bigg\rvert_{\lambda = 0} = \int f'(u(x)) v(x) \, dx = \langle f'(u), v \rangle.
\end{align*}
Since \eqref{eq:grad} must hold for any choice of $v$, we see that the claim is true.
\end{proof}


\textbf{Reference Materials.}
\begin{itemize}
\item \url{https://en.wikipedia.org/wiki/Calculus_of_variations}
\item \url{https://www2.math.uconn.edu/~gordina/NelsonAaronHonorsThesis2012.pdf}
\end{itemize}

\newpage

\subsection{Finding the Maximizer of $\elbo$} \label{append:varelbo}

\begin{itemize}
\item We solve the constrained optimization problem
\begin{align*}
\max_q \ \elbo[q] \quad \text{s.t.} \ \int q(\bm{z}) \, d\bm{z} = 1.
\end{align*}
\item To this end, we form the Lagrangian
\begin{align*}
L(q,\lambda) = \elbo[q] + \lambda \left( \int q(\bm{z}) \, d\bm{z} - 1 \right).
\end{align*}
\item The maximizer of $\elbo$ should satisfy the Lagrangian stationarity condition
\begin{align*}
\nabla L(q,\lambda) = 0.
\end{align*}
Using the definition of $\elbo$ and Proposition 1, we see that
\begin{align*}
\nabla L(q,\lambda) &= \nabla \left[ \elbo[q] + \lambda \left( \int q(\bm{z}) \, d\bm{z} - 1 \right) \right] \\
&= \nabla \left[ \EE_{q(\bm{z})}[\log p(\bm{z},\bm{x})] - \EE_{q(\bm{z})}[\log q(\bm{z})] + \lambda \left( \int q(\bm{z}) \, d\bm{z} - 1 \right) \right] \\
&= \nabla \left[ \int q(\bm{z}) \log p(\bm{z},\bm{x}) \, d\bm{z} - \int q(\bm{z}) \log q(\bm{z}) \, d\bm{z} + \lambda \left( \int q(\bm{z}) \, d\bm{z} - 1 \right) \right] \\
&= \nabla \int q(\bm{z}) \log p(\bm{z},\bm{x}) \, d\bm{z} - \nabla \int q(\bm{z}) \log q(\bm{z}) \, d\bm{z} + \lambda \nabla \int q(\bm{z}) \, d\bm{z} \\
&= \log p(\bm{z}, \bm{x}) - \log q(\bm{z}) - 1 + \lambda
\end{align*}
and so to satisfy the stationarity condition, we should have
\begin{align*}
q(\bm{z}) = p(\bm{z},\bm{x})e^{\lambda - 1}.
\end{align*}
\item Due to the optimization constraint,
\begin{align*}
1 = \int q(\bm{z}) \, d\bm{z} &= \int p(\bm{z},\bm{x})e^{\lambda - 1} \, d\bm{z} \\
&= e^{\lambda - 1} \int p(\bm{z},\bm{x}) \, d\bm{z} \\
&= e^{\lambda - 1} p(\bm{x})
\end{align*}
so we obtain
\begin{align*}
e^{\lambda - 1} = \frac{1}{p(\bm{x})}.
\end{align*}
\item It follows that
\begin{align*}
q(\bm{z}) = p(\bm{z},\bm{x})e^{\lambda - 1} = \frac{p(\bm{z},\bm{x})}{p(\bm{x})} = p(\bm{z} \mid \bm{x}).
\end{align*}
\end{itemize}

\textbf{Reference Material.}
\begin{itemize}
\item \url{https://wiki.inf.ed.ac.uk/twiki/pub/MLforNLP/WebHome/bkj-VBwalkthrough.pdf}
\end{itemize}
























































\end{document}