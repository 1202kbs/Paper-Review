\documentclass[10pt]{article}

\usepackage{mathtools}
\usepackage{fullpage}
\usepackage{latexsym}
\usepackage{csquotes}
\usepackage{enumitem}
\usepackage{mathrsfs}
\usepackage{hyperref}
\usepackage{tikz-cd}
\usepackage{amsmath}
\usepackage{amssymb}
\usepackage{gensymb}
\usepackage{parskip}
\usepackage{xcolor}
\usepackage{amsthm}
\usepackage{tikz}
\usepackage{bm}

\newenvironment{solution}{\textit{Solution.}}{\hfill$\square$}

\newcommand{\ZZ}{\mathbb{Z}}
\newcommand{\QQ}{\mathbb{Q}}
\newcommand{\RR}{\mathbb{R}}
\newcommand{\PP}{\mathbb{P}}
\newcommand{\EE}{\mathbb{E}}

\newcommand{\CC}{\mathcal{C}}
\newcommand{\FF}{\mathcal{F}}
\newcommand{\HH}{\mathcal{H}}
\newcommand{\KK}{\mathcal{K}}
\newcommand{\LL}{\mathcal{L}}
\newcommand{\MM}{\mathcal{M}}
\newcommand{\NN}{\mathcal{N}}
\newcommand{\OO}{\mathcal{O}}

\newcommand{\data}{\texttt{data}}
\newcommand{\noise}{\texttt{noise}}
\newcommand{\joint}{\texttt{joint}}
\newcommand{\infonce}{\texttt{InfoNCE}}
\newcommand{\flatnce}{\texttt{FlatNCE}}
\newcommand{\nce}{\texttt{NCE}}
\newcommand{\KL}{D_{\text{KL}}}
\newcommand{\JS}{D_{\text{JS}}}
\newcommand{\stopg}{\texttt{stop\char`_grad}}

\DeclareMathOperator*{\argmax}{arg\,max}
\DeclareMathOperator*{\argmin}{arg\,min}


\title{Paper Review: Breaking the log-K Curse on Contrastive Learners With FlatNCE}
\author{Beomsu Kim\footnote{Department of Mathematical Sciences, KAIST. Email \texttt{beomsu.kim@kaist.ac.kr}}}
\date{\today}

\begin{document}

\maketitle

\textbf{Paper Information.}

\begin{itemize}
\item Junya Chen et. al. Breaking the log-K Curse on Contrastive Learners With FlatNCE. \\ \underline{arXiv preprint arXiv:2107.01152}, 2021.
\end{itemize}

\section{Introduction}

\begin{itemize}
\item There are many unresolved issues with contrastive learning.
\begin{itemize}
\item Contrastive learners need a very large number of negative samples to work well.
\item The bias, variance, and performance tradeoffs are in debate.
\item There is a lack of training diagnostic tools for contrastive learning.
\end{itemize}
\item Our development starts with two simple intuitions.
\begin{itemize}
\item The contrasts between positive and negative data should be as large as possible.
\item The objective should be properly normalized to yield minimal variance.
\end{itemize}
\end{itemize}

\newpage

\section{Contrastive Representation Learning with InfoNCE}

\begin{itemize}
\item With $y_{1:K} = (y_1, \ldots, y_K)$, define
\begin{align*}
p_{XY^K}(x,y_{1:K}) = p_{XY}(x,y_1) \prod_{k \neq 1} p_Y(y_k).
\end{align*}
This means $(x,y_1) \sim p_{XY}$ and $(x,y_k) \sim p_X \otimes p_Y$ for $k \neq 1$.
\item Let $g(x,y)$ be a parametrized function, such as a neural network.
\item Given $(x,y_{1:K}) \sim p_{XY^N}$, define $g_k = g(x,y_k)$ for $k = 1, \ldots, K$. Also, define
\begin{align*}
I_\infonce^K(x,y_{1:K} \mid g) = \log \frac{\exp(g_1)}{\frac{1}{K} \sum_{k = 1}^K \exp(g_k) },
\end{align*}
and
\begin{align*}
I_\infonce^K(X;Y \mid g) = \EE_{p_{XY^K}} \left[ I_\infonce^K(x,y_{1:K} \mid g) \right]
\end{align*}
and
\begin{align*}
I_\infonce^K(X;Y) = \max_g \ I_\infonce^K(X;Y \mid g).
\end{align*}
Note that $g_1$ is the logit for the \enquote{positive pair} and $g_k$ for $k \neq 1$ are the logits for the \enquote{negative pairs}.
\item We also define
\begin{align*}
I_\infonce(x,y_{1:K} \mid g) = - \log \frac{\exp(g_1)}{\sum_{k = 1}^K \exp(g_k)}
\end{align*}
such that
\begin{align*}
I^K_\infonce(x,y_{1:K} \mid g) = -I_\infonce(x,y_{1:K} \mid g) + \log K.
\end{align*}
\end{itemize}

\textbf{Proposition 2.1.} $I_\infonce^K(X;Y)$ \textit{is an asymptotically tight lower bound to mutual information, i.e.,}
\begin{align*}
I(X;Y) \geq I_\infonce^K(X;Y \mid g), \qquad \lim_{K \rightarrow \infty} I_\infonce^K(X;Y) \rightarrow I(X;Y).
\end{align*}
\begin{proof}
See my paper review on InfoNCE.
\end{proof}

\newpage

\section{FlatNCE and Generalized Contrastive Representation Learning}

\begin{itemize}
\item Define
\begin{align*}
I_\flatnce(x,y_{1:K} \mid g) = \frac{\sum_{k \neq 1} \exp(g_k - g_1)}{\stopg[\sum_{k \neq 1} \exp(g_k - g_1)]}
\end{align*}
and
\begin{align*}
I^K_\flatnce(x,y_{1:K} \mid g) = - \log \frac{1}{K} \sum_{k \neq 1} \exp(g_k - g_1).
\end{align*}
\item We observe that
\begin{align*}
\sum_{k \neq 1} \exp(g_k - g_1) = \left( \frac{\exp(g_1)}{\sum_{k \neq 1} \exp(g_k)} \right)^{-1}
\end{align*}
and so (the gradient is w.r.t. the parameters of $g$)
\begin{align*}
\nabla I_\flatnce(x,y_{1:K} \mid g) &= \frac{\nabla \sum_{k \neq 1} \exp(g_k - g_1)}{\stopg[\sum_{k \neq 1} \exp(g_k - g_1)]} \\
&= \nabla \log \sum_{k \neq 1} \exp(g_k - g_1) \\
&= \nabla \log \frac{1}{K} \sum_{k \neq 1} \exp(g_k - g_1) \\
&= - \nabla I^K_\flatnce(x,y_{1:N} \mid g).
\end{align*}
This shows that gradient descent on $I_\flatnce$ is equivalent to gradient descent on $I^{\oplus,K}_\flatnce$.
\item We also observe that
\begin{align*}
I^K_\flatnce(x,y_{1:N} \mid g) &= - \log \frac{1}{K} \sum_{k \neq 1} \exp(g_k - g_1) \\
&= - \log \frac{\frac{1}{K} \sum_{k \neq 1} \exp(g_k)}{\exp(g_1)} \\
&= \log \frac{\exp(g_1)}{\frac{1}{K} \sum_{k \neq 1} \exp(g_k)}
\end{align*}
which is just $I_\infonce(x,y_{1:K} \mid g)$ with the positive pair logit $g_1$ removed from the denominator sum.

\newpage

\item Define
\begin{align*}
I^\oplus_\flatnce(x,y_{1:K} \mid g) = \frac{\sum_{k = 1}^K \exp(g_k - g_1)}{\stopg[\sum_{k = 1}^K \exp(g_k - g_1)]}
\end{align*}
and
\begin{align*}
I^{\oplus,K}_\flatnce(x,y_{1:K} \mid g) = - \log \frac{1}{K} \sum_{k = 1}^K \exp(g_k - g_1).
\end{align*}
\item Similar to $I_\flatnce$, we have
\begin{align*}
\nabla I^\oplus_\flatnce(x,y_{1:K} \mid g) = - \nabla I^{\oplus,K}_\flatnce(x,y_{1:K} \mid g)
\end{align*}
and also
\begin{align*}
I^{\oplus,K}_\flatnce(x,y_{1:K} \mid g) = \log \frac{\exp(g_1)}{\frac{1}{K} \sum_{k = 1}^K \exp(g_k) } = I_\infonce^K(x,y_{1:K} \mid g).
\end{align*}
which shows that gradient ascent on $I^\oplus_\flatnce$ is equivalent to gradient descent on $I_\infonce^K$.
\end{itemize}

\textbf{Proposition 3.1.} $\nabla I^\oplus_\flatnce(x,y_{1:K} \mid g) = \nabla I_\infonce(x,y_{1:K} \mid g)$.
\begin{proof}
Since (see Section 2)
\begin{align*}
I^K_\infonce(x,y_{1:K} \mid g) = -I_\infonce(x,y_{1:K} \mid g) + \log K,
\end{align*}
we have (see the above bullet)
\begin{align*}
\nabla I^\oplus_\flatnce(x,y_{1:K} \mid g) = - \nabla I^{\oplus,K}_\flatnce(x,y_{1:K} \mid g) = - \nabla I_\infonce^K(x,y_{1:K} \mid g) = \nabla I_\infonce(x,y_{1:K} \mid g).
\end{align*}
This concludes the proof.
\end{proof}

\textbf{Proposition 3.2.} \textit{The gradient of} $I_\flatnce(x,y_{1:K} \mid g)$ \textit{is an importance-weighted estimator of the form}
\begin{align*}
\nabla I_\flatnce(x,y_{1:K} \mid g) = \sum_{k \neq 1} w_k \nabla g_k - \nabla g_1, \qquad w_k = \frac{\exp(g_k)}{\sum_{k' \neq 1} \exp(g_{k'})}.
\end{align*}
\begin{proof}
Observe that
\begin{align*}
\nabla I_\flatnce(x,y_{1:K} \mid g) &= \frac{\nabla \sum_{k \neq 1} \exp(g_k - g_1)}{\sum_{k \neq 1} \exp(g_k - g_1)} \\
&= \sum_{k \neq 1} \frac{\exp(g_k - g_1)}{\sum_{k' \neq 1} \exp(g_{k'} - g_1)} (\nabla g_k - \nabla g_1) \\
&= \sum_{k \neq 1} \frac{\exp(g_k)}{\sum_{k' \neq 1} \exp(g_{k'})} (\nabla g_k - \nabla g_1) \\
&= \sum_{k \neq 1} w_k (\nabla g_k - \nabla g_1) \\
&= \sum_{k \neq 1} w_k \nabla g_k - \nabla g_1
\end{align*}
since $\sum_{k \neq 1} w_k = 1$.
\end{proof}

\newpage

\begin{itemize}
\item In FlatNCE, larger weights will be assigned to the more challenging negative examples in the batch.
\item The authors claim FlatNCE is also a formal MI lower bound using the below Lemma.
\end{itemize}

\textbf{Lemma 3.3.} \textit{For arbitrary $u \in \RR$, we have}
\begin{align*}
I^K_\infonce(x,y_{1:K} \mid g) \geq 1 - u - \frac{1}{K} \sum_{k = 1}^K \exp(-u + g_k - g_1)
\end{align*}
\textit{and the inequality holds when}
\begin{align*}
u = \stopg \left[ \log \frac{1}{K} \sum_{k = 1}^K \exp(g_k - g_1) \right] = \stopg \left[ -I^K_\infonce(x,y_{1:K} \mid g) \right].
\end{align*}
\begin{proof}
The Fenchel-Legendre dual for $f(t) = - \log t$ is $f^*(v) = -1 - \log(-v)$. That is,
\begin{align*}
f(t) = \sup_{v \in \RR} \ \{vt - f^*(v)\}
\end{align*}
and so
\begin{align*}
- \log t \geq vt + 1 + \log(-v)
\end{align*}
for any $v \in \RR$. Setting $v = -e^{-u}$, we get
\begin{align*}
- \log t \geq 1 - u - e^{-u} t
\end{align*}
for any $u \in \RR$. Since
\begin{align*}
I^K_\infonce(x,y_{1:K} \mid g) = I^{\oplus,K}_\flatnce(x,y_{1:K} \mid g) = - \log \frac{1}{K} \sum_{k = 1}^K \exp(g_k - g_1),
\end{align*}
setting
\begin{align*}
t = \frac{1}{K} \sum_{k = 1}^K \exp(g_k - g_1)
\end{align*}
proves the first part of the proposition. The second part can be checked by simple calculation.
\end{proof}

\textbf{Corollary 3.4.} $\EE_{p_{XY^N}} \left[ I^K_\flatnce(x,y_{1:K} \mid g) \right] \leq I(X;Y)$.
\begin{proof}
Plugging in the optimal value of $u$ in the above Lemma, we have
\begin{align*}
I^K_\infonce(x,y_{1:K} \mid g) = 1 + \stopg \left[ I^K_\infonce(x,y_{1:K} \mid g) \right] - I^\oplus_\flatnce(x,y_{1:K} \mid g).
\end{align*}
Since the first two terms at the RHS are constant w.r.t. the parameters of $g$, the authors claim that the claimed inequality holds up to a constant. However, the above inequality is about $I^\oplus_\flatnce$, not $I^K_\flatnce$, so I don't think the proof is correct.
\end{proof}

























\end{document}